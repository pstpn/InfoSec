\chapter{Технологический раздел}

\section{Средства реализации}
В качестве языка программирования, используемого при написании данной лабораторной работы, был выбран C++~\cite{cpp-lang}.
	
\section{Реализация алгоритмов}
На листингах~\ref{lst:des.hpp}~--~\ref{lst:cfb.hpp} представлены реализации разрабатываемых модулей.
\includelistingpretty
	{des.hpp}
	{c++}
	{Реализация класса алгоритма DES}

\includelistingpretty
	{cfb.hpp}
	{c++}
	{Реализация класса шифровального алгоритма DES в режиме работы CFB}

\section{Тестирование}
В таблице~\ref{tbl:functional_test} представлены функциональные тесты.
\begin{table}[ht!]
	\begin{center}
		\captionsetup{justification=raggedright,singlelinecheck=off}
		\caption{\label{tbl:functional_test} Функциональные тесты}
		\begin{tabular}{|m{1em}|m{15em}|m{15em}|}
			\hline
			№ & Входные данные & Выходные данные \\ 
			\hline
			1 & пустая строка & пустая строка \\
			\hline
			2 & пустой файл & пустой файл \\
			\hline
			3 & aaaaa & зашифрованный "aaaaa" \\
			\hline
			4 & зашифрованный "aaaaa" & aaaaa \\
			\hline
			5 & abcde & зашифрованный "abcde" \\
			\hline
			6 & зашифрованный "abcde" & abcde \\
			\hline
			7 & файл input.txt & зашифрованный файл input.txt \\
			\hline
			8 & зашифрованный файл input.txt & файл input.txt \\
			\hline
			9 & файл input.jpg & зашифрованный файл input.jpg \\
			\hline
			10 & зашифрованный файл input.jpg & файл input.jpg \\
			\hline
			11 & файл input.zip & зашифрованный файл input.zip \\
			\hline
			12 & зашифрованный файл input.zip & файл input.zip \\
			\hline
		\end{tabular}
	\end{center}
\end{table}
