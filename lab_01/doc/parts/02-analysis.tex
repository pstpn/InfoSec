\chapter{Аналитический раздел}

\section{Работа шифровальной машины}

Шифровальная машина <<Энигма>> появилась в 1919 году и выглядела как портативная печатная машинка, питаемая от батареи.
Оператор нажимал одну букву, и три зубчатых колеса ротора, которые менялись
ежедневно, преобразовывали эту букву в другую.
Далее электрические контакты создавали другие наборы перестановок, исходная буква изменялась семь раз, а потом загоралось светящееся окошко, имевшее вид буквы.
Второй оператор видел эту букву, а потом передавал их группами по пять с помощью азбуки Морзе~\cite{intro}.

Машина включала в себя четыре отсека: три служат для роторов и один - для расположения в нем рефлектора.
По своему строению ротор имел 26 сечений, по одному в соответствии каждой букве латинского алфавита; кроме этого в нем было 26 контактов, которые служат в качестве элементов соединения с другими роторами.
В то время как оператор нажимает на кнопку, цепь в шифровальной машине замыкается,  после чего появляется зашифрованная буква.
Цепь замыкалась также при помощи рефлектора, а реализация шифровальной машины имела ряд уникальных свойств~\cite{alg}:
\begin{enumerate}
	\item зашифрованные тексты симметричны: если установить одни и те же роторы в одном и том же порядке, то повторно закодированные сообщения будут одинаковы;
	\item при кодировании одинаковых и идущих друг за другом символов на выходе образуются абсолютно разные буквы;
	\item предыдущее свойство обуславливало невозможность совпадения исходного и зашифрованного символов.
\end{enumerate}

\section*{Вывод}

В аналитическом разделе был описан принцип работы шифровальной машины <<Энигма>>.