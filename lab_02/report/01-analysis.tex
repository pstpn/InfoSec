\chapter{Аналитический раздел}

\section{Алгоритм DES}

Шифровальная алгоритм DES~---~симметричный шифровальный алгоритм, разработанный в 1977 году компанией IBM. 
Он использует блочное шифрование, длина блока фиксирована и равна 64 битам. 
Однако каждые 8 бит в ключе игнорируются, что приводит к правильной длине ключа 56 бит в DES. 
Однако в любом случае один блок на 64 бита является вечной организацией DES.
Он состоит из 3 следующих шагов: 
\begin{itemize}
	\item[--] начальная перестановка, во время которой биты переставляются в порядке, определённом в специальной таблице;
	\item[--] 16 раундов шифрования;
	\item[--] завершающей перестановки, соовершающей преобразования, обратные сделанным на первом шаге.
\end{itemize}

Раунд шифрования состоит из 5 следующих этапов:
\begin{enumerate}[label=\arabic*)]
	\item расширение;
	\item получение ключа раунда;
	\item скремблирование;
	\item перестановка;
	\item смешивание ключа.
\end{enumerate}

Расширение, во время которого каждая из половин блока шифрования по 32 бит дополняется путём перестановки и дублировоания бит до длины в 48 бит.

Получение ключа раунда необходимо для применения в раунде шифрования 48-битного ключа раунда, полученного из основного ключа DES.
Основной ключ имеет длину 64 бита, однако значащих бит из 64 всего 56, остальные добавлены для избыточности и контроля передачи ключа.
Из этих 56 бит получают 48 путём разбиения на равные части и применению битовой операции циклического сдвига и нахождению нового значения посредством специальной таблицы.

Скремблирование предназначено для получения из 48-битного потока 32-битного путём разбиения на 6 частей по 8 бит и обработки каждой части в S-блоках, которые заменяют блоки с длиной 6 бит на блоки 4 бит посредством использования специальной таблицы.

Перестановка представляет из себя перемешивания полученной последовательности из 32 бит при помощи таблицы перемешивания.

Смешивание ключа представляет из себя операцию XOR полученного 32-битного значения c ключом раунда.

\section{Режимы работы алгоритма DES}

Режим шифрования~---~метод применения блочного шифра, позволяющий преобразовать последовательность блоков открытых данных в последовательность блоков зашифрованных данных.

Наиболее широкое распространение получили режимы~\cite{cpp-lang}:
\begin{enumerate}[label=\arabic*)]
	\item электронный шифроблокнот (Electronic Codebook ) - ECB;
	\item цепочка цифровых блоков (Cipher Block Chaining) - CBC;
	\item цифровая обратная связь (Cipher Feedback) - CFB;
	\item внешняя обратная связь (Output Feedback) - OFB.
\end{enumerate}

\clearpage

\section{CBF}

В этом режиме размер блока может отличаться от 64.
Исходный файл M считывается последовательными t-битовыми блоками (t <= 64): M = M(1)M(2)...M(n) (остаток дописывается нулями или пробелами).
64-битовый сдвиговый регистр (входной блок) вначале содержит вектор инициализации IV, выравненный по правому краю.
Для каждого сеанса шифрования используется новый IV.
Для всех i = 1...n блок шифртекста C(i) определяется следующим образом:

C(i) = M(i) xor P(i-1) ,

где P(i-1) - старшие t битов операции DES(С(i-1)), причем C(0)=IV.
Обновление сдвигового регистра осуществляется путем удаления его старших t битов и дописывания справа C(i).
Восстановление зашифрованных данных также не представляет труда: P(i-1) и C(i) вычисляются аналогичным образом и M(i) = C(i) xor P(i-1)~\cite{cpp-lang}.
